\section{Fazit}

Die aktive Gestaltung der Energiewende ist zwingend erforderlich um die Klimaziele zu erreichen.
Dies erfordert eine übergreifende Organisation, die alle Stakeholder einbezieht, und eine gemeinsame Strategie, welche insbesondere durch den Staat und die Gesellschaft getragen wird.
Dies erlaubt nicht nur einen schnellen und effektiven Wandel, es unterstützt die deutsche Wirtschaft, sichert unseren Wohlstand und Technologiestandort.
Entwickeln wir keine gemeinsame Strategie und keine integrierten Maßnahmen und Projekte, werden dies andere Länder tun was sich negativ auf unsere Wirtschaft auswirken wird.
Allein auf die Innovationsfreude der deutschen Wirtschaft zu setzen hat bisher zu keiner signifikanten CO\textsubscript{2}-Reduktion geführt.
Dies zeigt, dass die Wirtschaft ein klares staatliches Signal und eine klare Richtung braucht um die Entwicklung von Technologien, Produkten und Dienstleistungen anzugehen.
Genau hier setzt die Energiewendeagentur an und koordiniert Forschung und Entwicklung mit Wirtschaft, Gesellschaft, Universitäten und Forschungseinrichtungen.
