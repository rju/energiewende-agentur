\section{Rationale}

% - Warum ist das notwendig?
%   - Klimawandel
%   - Technologiewandel
%   - Gesteuerte und vernetzte Entwicklung 
Der Klimawandel und dessen vorhergesagten Folgen haben die globale Gemeinschaft dazu bewogen sich auf Gegenmaßnahmen zu verständigen. Als Obergrenze wurde eine maximale Erwärmung von 1.5 \degree{}C bzw. 2 \degree{}C vereinbart um die Folgen beherrschbar zu halten.
Daraus folgt eine globale Restmenge an Treibhausgasen, die noch in die Atmosphäre eingebracht werden kann, bevor wir -- die Menschheit -- die Folgen nicht mehr geeignet adressieren können.
%
Der \textbf{CO\textsubscript{2}-Ausstoß} muss deshalb radikal \textbf{reduziert} werden.
Dies erfordert neben einem \textbf{technologischen Wandel} in Verkehr, Industrie, Habitat und Landwirtschaft, einen \textbf{gesellschaftlichen Wandel}, da sich unsere Art zu arbeiten, reisen und produzieren ändern wird.
Der technologische Wandel erfordert eine Veränderung unserer Geschäftsmodelle insbesondere in zentralen Bereichen der Industrie.
Dies kann die Chance sein unsere Wirtschaft zukunftsfähig zu machen.
Gestalten wir den Wandel nicht, ist dies eine ernsthafte Bedrohung für unsere Ökonomie.

Diese Herausforderungen können nur gemeinsam durch Gesellschaft, Wirtschaft, Politik und Wissenschaft gelöst werden.
Dazu soll eine \textbf{inter- und transdisziplinäre Einrichtung} geschaffen werden, welche die Energiewende organisiert, Vorhaben der verschiedenen Institute, Einrichtungen und Ministerien koordiniert und integriert, und gesellschaftliche Akteure und Stakeholder einbindet.
Dabei ist es insbesondere wichtig die Bürger einzubinden und die Energiewende ihr Projekt werden zu lassen.
Andernfalls wird der Wandel, wie bei der Digitalisierung, als Bedrohung gesehen.

% - Warum es kein Ministerium sein soll?
%   - Querschnittsthema über mehrere Ministerien hinweg
An verschiedenen Stellen wurde bereits die Forderung erhoben ein Ministerium zu schaffen, welches sich ausschließlich um die Energiewende kümmert.
Dieser Ansatz hat jedoch drei Schwächen:
\begin{enumerate}
\item Ein solches Ministerium würde automatisch Kompetenzen anderer Ministerien einschränken oder übernehmen. Dies befördert klassische Verteilungskämpfe und erschwert die Kooperation.

\item Der Ansatz führt zu monothematischen Lösungsansätzen, welche dem transdisziplinären und transformatorischen Charakter nicht gerecht wird und wichtige Akteure und Multiplikatoren ausschließt.

\item Bestehende Ministerien, welche eine Querschnittsaufgabe haben, wie z.B. das Umweltministerium, zeigen, dass durch das Herauslösen und Ausgliedern von andere Ministerien diese Themen wenig beachten und inhaltliche Konflikte spät auf Ministerebene gelöst werden müssen.
\end{enumerate}

% - Warum kann dies nicht von anderen Einrichtungen geleistet werden?
Schon heute gibt es auf Bundesebene zahlreiche Einrichtungen, und Initiativen, welche sich um Aspekte der Umweltverträglichkeit und Energiewende kümmern.
Allerdings sind deren Aktivitäten kaum vernetzt, es fehlt an einer gemeinsamen Strategie und Bürgereinbindung.
So fördern z.B. DFG, BMWi und BMWF jeweils Entwicklungen im Bereich Speichertechnologien, es gibt jedoch keine gemeinsame Strategie und Kopplung aller Aspekte von der Grundlagenforschung über die Pilotierung und Produktentwicklung bis hin zur gesellschaftlichen Akzeptanz.

Da ein Ministerium als Struktur ungeeignet ist und die Integration der Initiativen über Einrichtungen hinweg unbefriedigend ist, bedarf es einer Organisation innerhalb der Regierung, welche auf Arbeitsebene integrativ tätig ist, eine gemeinsame Strategie mit allen Akteuren entwickelt und deren Umsetzung begleitet.
