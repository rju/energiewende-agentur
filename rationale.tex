\section{Rationale}

% - Warum ist das notwendig?
%   - Klimawandel
%   - Technologiewandel
%   - Gesteuerte und vernetzte Entwicklung 
Der Klimawandel und dessen potentiellen Folgen haben die globale Gemeinschaft dazu bewogen Gegenmaßnahmen zu ergreifen. Als Obergrenze wurde eine maximale Erwärmung von 1.5 \degree{}C bzw. 2 \degree{}C vereinbart um den Klimawandel und dessen Folgen beherrschbar zu halten.
Daraus ergibt sich eine globale Restmenge an Treibhausgasen, die noch in die Atmosphäre eingebracht werden kann, bevor wir -- die Menschheit -- die Folgen nicht mehr geeignet adressieren können.

Um dieses globale Ziel zu erreichen muss der CO2-Ausstoß radikal reduziert werden.
Dies erfordert neben einem technologischen Wandel in Verkehr, Industrie, Habitat und der Landwirtschaft, auch einen gesellschaftlichen Wandel, da sich u.a. unsere Art zu arbeiten, reisen und produzieren ändern wird.
Ferner fordert der technologische Wandel eine Veränderung der Geschäfts\-modelle gerade auch in zentralen Bereichen unserer Industrie.
Dies kann Chance sein um die Wirtschaft zukunftsfähig zu machen.
Handeln wir nicht, resultiert daraus eine ernsthafte Bedrohung für unsere Ökonomie.

Aus diesen Herausforderungen ergibt sich der Bedarf für eine inter- und transdisziplinäre Einrichtung, welche die Energiewende organisiert, Vorhaben der verschiedenen Institute, Einrichtungen und Ministerien koordiniert und integriert, und gesellschaftliche Akteure und Stakeholder einbezieht.

% - Warum es kein Ministerium sein soll?
%   - Querschnittsthema über mehrere Ministerien hinweg
An verschiedenen Stellen wurde bereits die Forderung erhoben ein Ministerium zu schaffen, welches sich ausschließlich um die Energiewende kümmert.
Dieser Ansatz hat jedoch drei Schwächen.
Erstens, würde ein solches Ministerium Kompetenzen von allen anderen Ministerien übernehmen was klassische Verteilungskämpfe befördert und damit die Kooperation erschwert.
Zweitens, führt der Ansatz dazu, dass die Energiewende monothematisch adressiert wird, welches dem transdisziplinären und transformatorischen Charakter nicht gerecht wird.
Drittens, zeigen andere Querschnittsministerien, wie z.B. das Umweltministerium, dass ein herauslösen und ausgliedern von Aspekten dazu führt, dass in den anderen Ministerien an diese Sachverhalte wenig gedacht wird und inhaltliche Konflikte spät auf Ministerebene gelöst werden müssen.

% - Warum kann dies nicht von anderen Einrichtungen geleistet werden?
Schon heute gibt es auf Bundesebene zahlreiche Einrichtungen, und Initiativen welche sich um Aspekte der Umweltverträglichkeit und Energiewende kümmern.
Allerdings sind diese kaum vernetzt und integriert.
Ferner fehlt ihnen eine gemeinsame Strategie.
So fördern z.B. DFG, BMWi und BMWF jeweils Entwicklungen im Bereich Speichertechnologien, es gibt jedoch keine gemeinsame Strategie und Kopplung aller Aspekte von der Grundlagenforschung über die Pilotierung und Produktentwicklung bis hin zur gesellschaftlichen Akzeptanz.

Da sowohl ein Ministerium als Struktur ungeeignet ist und die Integration der Initiativen über Einrichtungen hinweg unbefriedigend ist, bedarf es einer Organisation innerhalb der Regierung, welche auf Arbeitsebene integrativ tätig ist, eine gemeinsame Strategie mit allen Akteuren entwickelt und deren Umsetzung begleitet.
