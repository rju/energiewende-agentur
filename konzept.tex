\section{Agenturkonzept}

Das Konzept adressiert die Ziele aus vier Perspektiven:
(1) Die transdisziplinäre Ausrichtung bindet die Bevölkerung frühzeitig ein um realitätsnahe Lösungen zu erreichen und deren Akzeptanz zu erhöhen.
(2) Die Verzahnung von Forschung und Entwicklung fördert den Transfer von Grundlagen- und Anwendungsforschung hin zu marktreifen Produkten und Dienstleistungen.
(3) Speziell an die Wirtschaft gerichtete Programme unterstützen deren Transformation.
(4) Alle Vorhaben werden zudem offen unseren europäischen Partnern kommuniziert und mit EU-Programmen verzahnt. 

\subsection{Transdisziplinarität}

% - Kommunikation und Einbindung der Bevölkerung
% - Unterstützung der Legislative und Exekutive bei der
%   Schaffung geeigneter rechtlicher Rahmenbedingungen
Die Schaffung einer nachhaltigen Energieinfrastruktur und einer nachhaltigen Nutzung erfordert eine Anpassung der Lebens- und Arbeitsweisen in weiten Teilen der Bevölkerung.
Zudem müssen rechtliche Rahmenbedingungen an diese neuen Bedingungen angepasst werden.
Dies erfordert einerseits die Bevölkerung bei der Erarbeitung der Herausforderungen und deren Lösungen in Forschung, Entwicklung, Erprobung und Umsetzung einzubinden und andererseits die Legislative und Exekutive in den Prozess zu integrieren.
Unser Konzept setzt dafür auf transdisziplinäre Ansätze, welche wie z.B. Co-Design, die Einbindung zahlreicher Stakeholder in Forschungs- und Entwicklungsprozesse ermöglicht.
Ferner werde wir alle Maßnahmen und Entwicklungsschritte transparent kommunizieren, sodass Mitbürger, Behörden und andere Gruppen, welche nicht unmittelbar in den Prozess eingebunden sind, jederzeit den Stand der Prozesse einsehen können.
Als Kommunikationskonzept kann hierfür auf Projektportale, Tagebücher und Social-Media-Plattformen zurückgegriffen werden\footnote{Stadt-Regio-Tram Gmunden \url{http://www.stadtregiotram-gmunden.at}}\footnote{Stuttgart-Ulm Bauportal \url{http://www.bahnprojekt-stuttgart-ulm.de}}.

% inter-ministerielle Kommunikation

\subsection{Integrierte Forschung und Entwicklung}

% - Integration von Forschungsförderprogrammen
% - Entwicklung von Forschungsförderprogrammen
% - Einbindung der industriellen Entwicklung
Es gibt in Deutschland bereits gut funktionierende Forschungsförderinstrumente und -programm, die jedoch voneinander unabhängig Themen adressieren und Vorhaben fördern.
Hier soll die EWA die Kommunikation zwischen den Forschungsförderern verbessern und mit diesen gemeinsam eine einheitliche Strategie entwickeln sowie deren Umsetzung begleiten.
Neben staatlichen und gemeinnützigen Förderern soll auch die industrielle Entwicklung angebunden werden, sodass Projekte schnell in Produkte und Dienstleistungen einfließen können.
Denn ohne eine aktive Einbindung der Wirtschaft wird die Transformation nicht erfolgreich sein und andere Nationen werden Europa und Deutschland hinter sich lassen.

\subsection{Einbindung der Wirtschaft und Wirtschaftsförderung}

% - Durchführung und Steuerung von Pilotprojekten
% - Betreuung und Unterstützung bei nationaler Ausbreitung von erfolgreichen Piloten
Für eine erfolgreiche Umsetzung von erarbeiteten Konzepten für die Energiewende ist die Pilotierung ein zentrales Instrument.
Dies gilt besonders auch für konkurrierende Konzepte und Technologien zur Umsetzung dieser Konzepte.
Hier wird die EWA die Ausschreibung, Koordinierung und Evaluation der Piloten steuern und dabei auf Begutachtungsinstrumente bestehender Forschungs- und Wirtschaftsfördereinrichtungen zurückgreifen.

Nach erfolgreicher Pilotierung und Auswahl von Technologien und Konzepten wird die EWA deren Ausrollen auf nationaler Ebene begleiten und unterstützen.

% nutzung von shared patents, RAND

\subsection{Europäischer Austausch und Integration}

% - Europäische Integration und Kommunikation von Ergebnissen
Die Energiewende ist eine globale Aufgabe, die der Koordinierung und des Austausches über Grenzen hinweg erfordert.
Insbesondere müssen technologische und organisatorische Konzepte und Maßnahmen mit unseren Nachbarn abgestimmt und integriert werden.

Die EWA wird den interministeriellen Austausch auf fachlicher Ebene befördern und dies primär thematisch zu organisieren.
Das bedeutet, die Koordinierung erfolgt nicht entlang ministeriellen Grenzen sondern sind immer ministerienübergreifend und auf ein konkretes Projektziel ausgerichtet.
$ beispiel

